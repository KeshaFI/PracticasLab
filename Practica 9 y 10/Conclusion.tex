La práctica realizada permitió alcanzar de manera integral los objetivos planteados, fortaleciendo la comprensión y aplicación de los principios de la programación orientada a objetos en el contexto de un sistema de gestión de servicios para un taller mecánico. Se definió un contrato claro mediante la interfaz \textit{ServicioTaller} (con \textbf{calcularServicio()} y \textbf{generarReporteServicio()}), y se organizó la jerarquía alrededor de la clase \textit{abstracta} \textbf{Vehiculo} y las clases concretas \textbf{Auto}, \textbf{Moto} y \textbf{Camion}. Este diseño aseguró coherencia y reutilización del comportamiento común, mientras que la herencia y la sobrescritura evidenciaron el \textbf{polimorfismo} a través de una misma interfaz de uso.

Adicionalmente, se reforzó el \textbf{encapsulamiento} de atributos mediante \textit{getters} y \textit{setters}, y se documentó la estructura con diagramas UML, facilitando la comunicación técnica y la mantenibilidad. La organización del código mediante el \textit{Full Qualified Name} propuesto contribuyó a la \textbf{modularidad} y a una estructura clara del proyecto.

En conjunto, los resultados muestran que el modelo propuesto es consistente y extensible: permitió registrar distintos tipos de vehículos, calcular costos de servicio y generar reportes desde el menú de la aplicación. En síntesis, la práctica consolidó \textbf{abstracción}, \textbf{herencia}, \textbf{polimorfismo}, \textbf{encapsulación} y \textbf{modularidad} como fundamentos para escalar el sistema sin comprometer su cohesión ni su claridad.
