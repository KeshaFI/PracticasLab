En el desarrollo de aplicaciones orientadas a objetos, es fundamental comprender cómo se estructuran las clases, cómo se relacionan entre sí y cómo se gestionan los errores que pueden surgir durante la ejecución. El archivo main.dart presenta una aplicación de consola escrita en Dart (usando Flutter como entorno base), que simula el funcionamiento de un sistema para la gestión de servicios en un taller mecánico. Esta aplicación incluye una jerarquía de clases que modela distintos tipos de vehículos (autos, motos y camiones), cada uno con sus propias características y lógica de cálculo de servicio, además de una interfaz común que garantiza la coherencia funcional.

\subsection{Planteamiento del problema}
La comprensión y mantenimiento del código puede volverse complejo si no se cuenta con una representación clara de su arquitectura y comportamiento. Además, el manejo de errores mediante excepciones está presente en varios puntos críticos del sistema, pero requiere ser analizado para entender su impacto en la robustez de la aplicación. Por ello, se realizara un reporte técnico que documente y analice los aspectos estructurales y dinámicos del sistema, así como el uso de excepciones, con el fin de mejorar la comprensión del diseño.

\subsection{Motivación}
Este ejercicio tiene como propósito fortalecer las habilidades de modelado y análisis en programación orientada a objetos, utilizando herramientas como los diagramas UML para representar tanto la estructura estática (clases, atributos, relaciones) como el comportamiento dinámico (interacciones, flujos de ejecución) del sistema. Además, se busca reflexionar sobre el uso de excepciones como mecanismo de control de errores, evaluando su implementación en el código y su contribución a la estabilidad del programa. Finalmente, se exige que la aplicación esté correctamente empaquetada bajo el nombre calificado mx.unam.fi.poo.p910.

\subsection{Objetivos}
\begin{itemize}
    \item Diagramas UML al menos uno estático y uno dinámico.
    \item Dar una interpretación de la aplicación del tema de excepciones dentro del código.
    \item La aplicación debe estar empaquetada con el Full Qualified Name mx.unam.fi.poo.p910.
\end{itemize}
