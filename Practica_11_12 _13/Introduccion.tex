En el estudio de la programación  no solo se requiere de la comprensión de los lenguajes y sus sintaxis, sino también la capacidad de analizar cómo los distintos conceptos se integran en el diseño y funcionamiento de un sistema. En este sentido, la práctica sobre archivos, hilos y patrones de diseño busca ofrecer una visión integral de tres componentes esenciales en el desarrollo de software, apoyándose en representaciones gráficas mediante diagramas UML estáticos y dinámicos y en la interpretación crítica de ejemplos concretos.

\subsection{Planteamiento del problema}

En el desarrollo de software, los estudiantes se deben enfrentar al reto de comprender y aplicar conceptos que van más allá de la programación básica. El manejo de archivos permite la persistencia y organización de datos; los hilos introducen la complejidad de la ejecución concurrente y la sincronización de procesos; mientras que los patrones de diseño ofrecen soluciones estructuradas a problemas recurrentes en la arquitectura de sistemas. Sin embargo, la dificultad radica en integrar estos elementos de manera coherente, comprendiendo no solo su implementación técnica, sino también su impacto en la calidad, eficiencia y mantenibilidad del código.

\subsection{Motivación}

La práctica surge de la necesidad de fortalecer las competencias en programación avanzada mediante ejemplos concretos que permitan visualizar cómo los conceptos teóricos se aplican en situaciones reales. El uso de diagramas UML estáticos y dinámicos se convierte en una herramienta clave para representar tanto la estructura como el comportamiento del sistema, facilitando la interpretación y el análisis crítico. Además, trabajar con material audiovisual proporcionado por el docente asegura un aprendizaje contextualizado, donde los estudiantes pueden relacionar la teoría con casos prácticos y desarrollar una visión más integral del diseño de software.

\subsection{Objetivos}

Se realizara un reporte con los siguientes temas:
\begin{itemize}
    \item Explicación de la aplicación del tema de Archivos en el ejemplo principal del video.
    \item Explicación de la aplicación del tema de Hilos en los ejemplos del video.
    \item Explicación de la aplicación del tema de Patrones de Diseño en el ejemplo principal del video.
\end{itemize}

Con base en los códigos, se reportara lo siguiente:
\begin{itemize}
    \item Diagramas UML estático.
    \item Diagramas UML dinámico.
    \item Interpretación de conceptos aplicados en el código.
\end{itemize}


