A continuación se pasará a explicar detalladamente cada parte fundamental de los temas en el siguiente orden:

\subsection{Archivos}
El programa proporcionado presenta un menú interactivo que permite al usuario realizar tres operaciones fundamentales sobre archivos de texto: crear y escribir, leer, y sobrescribir. Cada una de estas acciones está encapsulada en funciones específicas que reflejan buenas prácticas de modularidad y control de flujo.

\begin{itemize}
    \item 1.Crear y escribir archivos:
    La función crearYEscribirArchivo() permite al usuario crear un archivo nuevo y guardar texto línea por línea. Este proceso refleja el concepto de persistencia de datos y buffer manual.
    \begin{center}
    \begin{lstlisting}[
    % Opciones para la apariencia:
    frame=single,    % Dibuja el recuadro (el marco)
    numbers=left     % Muestra la numeración a la izquierda
    % Puedes añadir las opciones de centrado si es necesario:
    % xleftmargin=0.1\textwidth,
    % xrightmargin=0.1\textwidth
    ]
    File archivo = File(nombreArchivo);
    archivo.writeAsStringSync(lineas.join('\n'));

    \end{lstlisting}
    \end{center}
    \begin{itemize}
        \item Se crea una instancia de File, que representa un archivo en el sistema.

        \item writeAsStringSync escribe el contenido de forma sincrónica, garantizando que el archivo se guarde antes de continuar.

        \item El uso de lineas.join(\verb|'\n'|) simula un buffer: se acumulan las líneas en memoria antes de escribirlas.
    \end{itemize}
    \item 2.Leer archivos existentes: La función \verb|_|leerArchivoExistente() permite recuperar el contenido de un archivo previamente creado. Aquí se aplica el concepto de extracción de datos persistentes.
    \begin{center}
    \begin{lstlisting}[
    % Opciones para la apariencia:
    frame=single,    % Dibuja el recuadro (el marco)
    numbers=left     % Muestra la numeración a la izquierda
    % Puedes añadir las opciones de centrado si es necesario:
    % xleftmargin=0.1\textwidth,
    % xrightmargin=0.1\textwidth
    ]
    if (!archivo.existsSync()) {
      print('\nEl archivo no existe.\n');
      return;
    }
    
    String contenido = archivo.readAsStringSync();


    \end{lstlisting}
    \end{center}
    \begin{itemize}
        \item Se verifica la existencia del archivo con existsSync(), lo que evita errores de lectura.
        \item readAsStringSync() carga todo el contenido en una sola operación, útil para archivos pequeños.
    \end{itemize}
    \item 3.Sobrescribir archivos: La función \verb|_|sobrescribirArchivo() introduce el concepto de modificación destructiva. Antes de sobrescribir, se solicita confirmación explícita del usuario.
    \begin{center}
    \begin{lstlisting}[
    % Opciones para la apariencia:
    frame=single,    % Dibuja el recuadro (el marco)
    numbers=left     % Muestra la numeración a la izquierda
    % Puedes añadir las opciones de centrado si es necesario:
    % xleftmargin=0.1\textwidth,
    % xrightmargin=0.1\textwidth
    ]
    if (confirmacion == null || confirmacion.toUpperCase() != 'SI') {
    print('\nOperacion cancelada.\n');
    return;
    }



    \end{lstlisting}
    \end{center}
    \begin{itemize}
        \item Se protege la integridad del archivo solicitando confirmación.
        \item Si el usuario acepta, se recopila nuevo contenido y se usa writeAsStringSync para reemplazar el archivo.
    \end{itemize}
    \item Conexión entre funciones: Todas las funciones comparten un patrón de interacción:
    \begin{center}
    \begin{lstlisting}[
    % Opciones para la apariencia:
    frame=single,    % Dibuja el recuadro (el marco)
    numbers=left     % Muestra la numeración a la izquierda
    % Puedes añadir las opciones de centrado si es necesario:
    % xleftmargin=0.1\textwidth,
    % xrightmargin=0.1\textwidth
    ]
  String? entrada = stdin.readLineSync();



    \end{lstlisting}
    \end{center}
    \begin{itemize}
        \item Se usa entrada estándar para interactuar con el usuario, lo que permite construir un flujo conversacional.
        \item Este patrón refuerza la idea de que el usuario es parte activa del proceso de persistencia.
    \end{itemize}
\end{itemize}
El diagrama estático que representa el código Archivos es el siguiente:
\begin{figure}[H] % El entorno figure permite manejar la posición
    \centering % Centra la imagen
    \includegraphics[width=0.4\textwidth]{archivoestatico.png} % Ajusta el tamaño y coloca el nombre del archivo
    \caption{Diagrama estático / ARCHIVOS} % Texto debajo de la imagen
    \label{fig:ejemplo} % Etiqueta para referencias cruzadas
\end{figure}
El diagrama estático representa la estructura del programa en Dart que gestiona operaciones con archivos mediante un menú interactivo; la función principal main() actúa como punto de entrada y dirige el flujo hacia tres funciones privadas agrupadas conceptualmente en la clase File menu: \verb|_|crearYEscribirArchivo(), \verb|_|leerArchivoExistente() y \verb|_|sobrescribirArchivo(), cada una encargada de una operación específica sobre archivos de texto; estas funciones interactúan directamente con la clase File de Dart, utilizando sus métodos writeAsStringSync(), readAsStringSync() y existsSync() para realizar escritura, lectura y verificación de archivos de forma sincrónica,

El diagrama dinámico que representa el código Archivos es el siguiente:

\begin{figure}[H] % El entorno figure permite manejar la posición
    \centering % Centra la imagen
    \includegraphics[width=0.7\textwidth]{DiagramaDinamicoArchivos.png} % Ajusta el tamaño y coloca el nombre del archivo
    \caption{Diagrama dinámico / ARCHIVOS} % Texto debajo de la imagen
    \label{fig:ejemplo} % Etiqueta para referencias cruzadas
\end{figure}

El diagrama muestra cómo interactúa el usuario con tu programa, y qué métodos se ejecutan dependiendo de la opción que elija en el menú.

Hay 4 participantes:
\begin{itemize}
    \item Usuario (main) - es quien elige la opción.
    \item crearYEscribirArchivo()
    \item leerArchivoExistente()
    \item sobrescribirArchivo()
\end{itemize}

Cada uno representa un flujo posible cuando el usuario selecciona del 1 al 4.

\subsection{Hilos}
