\subsection{Archivos}
\begin{center}
    \includegraphics[width=0.8\textwidth]{Res1.png}
\end{center}
En la imagen anterior podemos observar el menú de la aplicación donde primero usamos la opción 1 para crear un archivo y escribir en este, para después usar la opción 2 para leer lo que escibimos confirmando que se escirbió bien.

\begin{center}
    \includegraphics[width=0.8\textwidth]{Res2.png}
\end{center}
En esta imagen podemos observar el uso de la opción 3 del menú, en la cual sobreescribimos el archivo eliminando lo antes escrito, volvemos a usar la opción 2 para comprobar que se sobreescribió correctamente. Para finalizar usamos la opción 4 para salir de la aplicación

\subsection{Hilos}
\begin{center}
    \includegraphics[width=0.8\textwidth]{Res3.png}
\end{center}
Se muestra como es que el método future permite esperar otra tarea para que la otra tarea asincrona pueda ocurrir.
Se ve en la terminal la primer el fin de la primer tarea y luego de dos segundos el fin de la segunda.



\begin{center}
    \includegraphics[width=0.8\textwidth]{Res4.png}
\end{center}
En esta imagen se muestra como await afecta el tiempo y el modo de ejecución de un bloque de código.


\begin{center}
    \includegraphics[width=0.8\textwidth]{Res5.png}
\end{center}
En la imagen se muestra que se mando un paso de mensajes de una tarea diferente mediante los puertos.


\begin{center}
    \includegraphics[width=0.8\textwidth]{Res6.png}
\end{center}
Se muestra en la imagen como con el uso de isolate repartimos las tareas, mientras un hilo se encarga de las impresiones otro se encarga de hacer el cálculo.


\begin{center}
    \includegraphics[width=0.8\textwidth]{Res7.png}
\end{center}
En esta imagen podemos apreciar como mientras hay un puerto que recibe mensajes hay otro que los manda. Un hilo se encargará de mandar y otro de recibir. Se podría interpretar como una especie de chat primitivo entre hilos.


\begin{center}
    \includegraphics[width=0.8\textwidth]{Res8.png}
\end{center}
Se muestra el uso de una suma serial sin repartir las tareas entre hilos.


\subsection{Patrones}
\begin{center}
    \includegraphics[width=0.6\textwidth]{Res9.png}
    \includegraphics[width=0.6\textwidth]{Res10.png}
\end{center}

Se muestra en la imagen la simulación de un combate pokemon primero se muestran los pokemons y sus características, para después mostrar la simulación del combate entre estos, esta simulación se detiene cuando la vida de uno llega a cero haciendo vencedor al pokemon con vida.


\begin{center}
    \includegraphics[width=0.6\textwidth]{Res11.png}
\end{center}

Se muestra el funcionamiento de la app, que es una simulación de una impresora.
Primero muestra los archivos que se mandaron a imprimir, así támbien muestra la cola de los archivos en impresión para después mostrar la simulación de la impresión de cada archivo y al final mostrar un historial de la impresora.



