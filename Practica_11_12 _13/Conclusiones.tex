
La práctica desarrollada permitió integrar de manera efectiva tres componentes esenciales en el diseño de software: el manejo de archivos como medio de persistencia de datos, la programación concurrente mediante hilos para optimizar el rendimiento, y la aplicación de patrones de diseño como guía estructural para la construcción de sistemas robustos y escalables.

A través del análisis de ejemplos concretos y la elaboración de los diagramas UML estáticos y dinámicos, se logró visualizar con claridad tanto la arquitectura del sistema como su comportamiento en tiempo de ejecución. Esta representación gráfica facilitó la interpretación de los conceptos aplicados en el código, permitiendo identificar relaciones, dependencias y flujos de ejecución que no siempre son evidentes en la lectura directa del programa.

Más allá de la implementación técnica, el enfoque de la práctica se centró en la interpretación crítica de los conceptos, promoviendo una comprensión profunda de las decisiones de diseño y su impacto en la calidad del software. Este ejercicio fortaleció las habilidades de análisis y modelado, pero también fomentó una actitud reflexiva frente al desarrollo, donde cada componente se evalúa en función de su propósito, eficiencia y coherencia dentro del sistema.

Por ende, la práctica contribuyó a consolidar una visión integral del desarrollo de software, articulando teoría y aplicación en un entorno de aprendizaje guiado por ejemplos y representaciones formales.
