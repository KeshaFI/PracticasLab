


\subsection{Archivo y dart:io}
La biblioteca dart:io proporciona API para gestionar archivos, directorios, procesos, sockets, WebSockets y clientes y servidores HTTP. \cite{dart_language}

En general, la biblioteca dart:io implementa y promueve una API asíncrona. Los métodos síncronos pueden bloquear fácilmente una aplicación, dificultando su escalabilidad. Por lo tanto, la mayoría de las operaciones devuelven resultados mediante objetos Future o Stream, un patrón común en plataformas de servidor modernas como Node.js.

Los pocos métodos sincrónicos de la biblioteca dart:io están claramente marcados con el sufijo Sync en el nombre del método. No se tratan aquí los métodos sincrónicos.
\begin{center}
\begin{lstlisting}[
    % Opciones para la apariencia:
    frame=single,    % Dibuja el recuadro (el marco)
    numbers=left     % Muestra la numeración a la izquierda
    % Puedes añadir las opciones de centrado si es necesario:
    % xleftmargin=0.1\textwidth,
    % xrightmargin=0.1\textwidth
]
import 'dart:io';
\end{lstlisting}
\end{center}



A Filecontiene una ruta donde se pueden realizar operaciones. Puedes obtener el directorio padre del archivo usando parent , una propiedad heredada de FileSystemEntity .

Cree un nuevo Fileobjeto con una ruta para acceder al archivo especificado en el sistema de archivos desde su programa.
\begin{center}
\begin{lstlisting}[
    % Opciones para la apariencia:
    frame=single,    % Dibuja el recuadro (el marco)
    numbers=left     % Muestra la numeración a la izquierda
    % Puedes añadir las opciones de centrado si es necesario:
    % xleftmargin=0.1\textwidth,
    % xrightmargin=0.1\textwidth
]
var myFile = File('file.txt');
\end{lstlisting}
\end{center}
La Fileclase contiene métodos para manipular archivos y su contenido. Con los métodos de esta clase, puede abrir y cerrar archivos, leerlos y escribir en ellos, crearlos y eliminarlos, y comprobar su existencia.

Al leer o escribir un archivo, puede utilizar secuencias (con openRead ), operaciones de acceso aleatorio (con open ) o métodos convenientes como readAsString.

La mayoría de los métodos de esta clase se ejecutan en pares síncronos y asincrónicos; por ejemplo, readAsString y readAsStringSync . A menos que tenga una razón específica para usar la versión síncrona de un método, prefiera la versión asíncrona para evitar bloquear su programa.
Si path es un enlace simbólico, en lugar de un archivo, entonces los métodos de Fileoperan en el destino final del enlace, excepto delete y deleteSync , que operan en el enlace.\cite{dart_file_class}
\begin{itemize}
    \item Leer desde un archivo: El siguiente ejemplo de código lee todo el contenido de un archivo como una cadena utilizando el método asincrónico readAsString 

    \begin{center}
    \begin{lstlisting}[
    % Opciones para la apariencia:
    frame=single,    % Dibuja el recuadro (el marco)
    numbers=left     % Muestra la numeración a la izquierda
    % Puedes añadir las opciones de centrado si es necesario:
    % xleftmargin=0.1\textwidth,
    % xrightmargin=0.1\textwidth
]
import 'dart:async';
import 'dart:io';

void main() {
  File('file.txt').readAsString().then((String contents) {
    print(contents);
  });
}
\end{lstlisting}
\end{center}

Una forma más flexible y útil de leer un archivo es con un Stream . Abra el archivo con openRead , que devuelve un stream que proporciona los datos del archivo como fragmentos de bytes. Lea el stream para procesar el contenido del archivo cuando esté disponible. Puede usar varios transformadores sucesivamente para manipular el contenido del archivo al formato requerido o para prepararlo para la salida.

Es posible que desee utilizar una secuencia para leer archivos grandes, para manipular los datos con transformadores o para lograr compatibilidad con otra API, como WebSockets .\cite{dart_file_class}
\begin{center}
\begin{lstlisting}[
    % Opciones para la apariencia:
    frame=single,    % Dibuja el recuadro (el marco)
    numbers=left     % Muestra la numeración a la izquierda
    % Puedes añadir las opciones de centrado si es necesario:
    % xleftmargin=0.1\textwidth,
    % xrightmargin=0.1\textwidth
]
import 'dart:io';
import 'dart:convert';
import 'dart:async';

void main() async {
  final file = File('file.txt');
  Stream<String> lines = file.openRead()
    .transform(utf8.decoder)       // Decode bytes to UTF-8.
    .transform(LineSplitter());    // Convert stream to individual lines.
  try {
    await for (var line in lines) {
      print('$line: ${line.length} characters');
    }
    print('File is now closed.');
  } catch (e) {
    print('Error: $e');
  }
}
\end{lstlisting}
\end{center}
\item Escribir en un archivo: Para escribir una cadena en un archivo, utilice el método writeAsString 
\begin{center}
\begin{lstlisting}[
    % Opciones para la apariencia:
    frame=single,    % Dibuja el recuadro (el marco)
    numbers=left     % Muestra la numeración a la izquierda
    % Puedes añadir las opciones de centrado si es necesario:
    % xleftmargin=0.1\textwidth,
    % xrightmargin=0.1\textwidth
]
import 'dart:io';

void main() async {
  final filename = 'file.txt';
  var file = await File(filename).writeAsString('some content');
  // Do something with the file.
}
\end{lstlisting}
\end{center}
También puedes escribir en un archivo usando un Stream . Abre el archivo con openWrite , que devuelve un IOSink donde puedes escribir datos. Asegúrate de cerrar el IOSink con el método IOSink.close .
\begin{center}
\begin{lstlisting}[
    % Opciones para la apariencia:
    frame=single,    % Dibuja el recuadro (el marco)
    numbers=left     % Muestra la numeración a la izquierda
    % Puedes añadir las opciones de centrado si es necesario:
    % xleftmargin=0.1\textwidth,
    % xrightmargin=0.1\textwidth
]
import 'dart:io';

void main() async {
  var file = File('file.txt');
  var sink = file.openWrite();
  sink.write('FILE ACCESSED ${DateTime.now()}\n');
  await sink.flush();

  // Close the IOSink to free system resources.
  await sink.close();
}
\end{lstlisting}
\end{center}
Para evitar bloqueos involuntarios del programa, varios métodos son asíncronos y devuelven un Future . Por ejemplo, el método length , que obtiene la longitud de un archivo, devuelve un Future . Espere a que el future obtenga el resultado cuando esté listo.
\begin{center}
\begin{lstlisting}[
    % Opciones para la apariencia:
    frame=single,    % Dibuja el recuadro (el marco)
    numbers=left     % Muestra la numeración a la izquierda
    % Puedes añadir las opciones de centrado si es necesario:
    % xleftmargin=0.1\textwidth,
    % xrightmargin=0.1\textwidth
]
import 'dart:io';

void main() async {
  final file = File('file.txt');

  var length = await file.length();
  print(length);
}
\end{lstlisting}
\end{center}
\end{itemize}
\subsection{Manejo de errores}
Su código Dart puede generar y capturar excepciones. Las excepciones son errores que indican que ocurrió algo inesperado. Si no se captura la excepción, el aislamiento que la generó se suspende y, por lo general, el aislamiento y su programa se cierran.

A diferencia de Java, todas las excepciones de Dart son excepciones no comprobadas. Los métodos no declaran qué excepciones podrían lanzar, y no es necesario capturar ninguna.\cite{dart_error_handling}

Dart proporciona tipos Exceptiony Error, así como numerosos subtipos predefinidos. Por supuesto, puede definir sus propias excepciones. Sin embargo, los programas Dart pueden lanzar cualquier objeto no nulo como excepción, no solo objetos de excepción y error.
\begin{center}
\begin{lstlisting}[
    % Opciones para la apariencia:
    frame=single,    % Dibuja el recuadro (el marco)
    numbers=left     % Muestra la numeración a la izquierda
    % Puedes añadir las opciones de centrado si es necesario:
    % xleftmargin=0.1\textwidth,
    % xrightmargin=0.1\textwidth
]
throw FormatException('Expected at least 1 section');
\end{lstlisting}
\end{center}

Capturar una excepción impide que se propague (a menos que se vuelva a lanzar). Capturar una excepción permite gestionarla:
\begin{center}
\begin{lstlisting}[
    % Opciones para la apariencia:
    frame=single,    % Dibuja el recuadro (el marco)
    numbers=left     % Muestra la numeración a la izquierda
    % Puedes añadir las opciones de centrado si es necesario:
    % xleftmargin=0.1\textwidth,
    % xrightmargin=0.1\textwidth
]
try {
  breedMoreLlamas();
} on OutOfLlamasException {
  buyMoreLlamas();
}
\end{lstlisting}
\end{center}
Para gestionar código que puede generar más de un tipo de excepción, se pueden especificar varias cláusulas catch. La primera cláusula catch que coincida con el tipo del objeto generado gestiona la excepción. Si la cláusula catch no especifica un tipo, puede gestionar cualquier tipo de objeto generado: 
\begin{center}
\begin{lstlisting}[
    % Opciones para la apariencia:
    frame=single,    % Dibuja el recuadro (el marco)
    numbers=left     % Muestra la numeración a la izquierda
    % Puedes añadir las opciones de centrado si es necesario:
    % xleftmargin=0.1\textwidth,
    % xrightmargin=0.1\textwidth
]
try {
  breedMoreLlamas();
} on OutOfLlamasException {
  // A specific exception
  buyMoreLlamas();
} on Exception catch (e) {
  // Anything else that is an exception
  print('Unknown exception: $e');
} catch (e) {
  // No specified type, handles all
  print('Something really unknown: $e');
}
\end{lstlisting}
\end{center}

Como se muestra en el código anterior, puede usar uno de los dos on, catcho ambos. Úselo oncuando necesite especificar el tipo de excepción. Úselo catchcuando su manejador de excepciones necesite el objeto de excepción.

Puedes especificar uno o dos parámetros para catch(). El primero es la excepción lanzada y el segundo es el seguimiento de la pila (un StackTraceobjeto).
\subsection{Programación asincrónica: futuros, asíncrono, espera}

Las operaciones asincrónicas permiten que su programa complete su trabajo mientras espera que finalice otra operación. Estas son algunas operaciones asincrónicas comunes:

Obteniendo datos a través de una red.
Escribiendo en una base de datos.
Leyendo datos de un archivo.
Estos cálculos asincrónicos suelen proporcionar su resultado como un Futureo, si el resultado tiene varias partes, como un Stream. Estos cálculos introducen asincronía en un programa. Para adaptar esta asincronía inicial, otras funciones Dart simples también deben volverse asincrónicas.

Para interactuar con estos resultados asincrónicos, puede usar las palabras clave " asyncy await". La mayoría de las funciones asincrónicas son simplemente funciones asíncronas de Dart que dependen, posiblemente en el fondo, de un cálculo inherentemente asincrónico.
\begin{itemize}
    \item 
\end{itemize}
\end{document}