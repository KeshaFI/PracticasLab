

El desarrollo de esta aplicación nos permitió integrar de manera práctica los conceptos fundamentales revisados durante el curso, particularmente aquellos relacionados con la programación orientada a objetos, la lógica estructurada y el diseño de interfaces en Flutter. A través de la implementación del sistema de batalla, se comprobó que es posible modelar comportamientos complejos como turnos, ataques, relaciones de tipos y variación de estadísticas utilizando clases, métodos y estructuras bien definidas. Esto demuestra el valor de las bases teóricas del curso al aplicarse directamente en el diseño de un sistema interactivo.

La simulación de batallas Pokémon desarrollada cumplió con los requisitos planteados inicialmente: se logró integrar un sistema de turnos basado en la velocidad de los Pokémon, una lógica de daño que respeta ventajas y desventajas de tipos, y un menú de acciones que permite ejecutar ataques o utilizar objetos estratégicamente. La correcta actualización de la interfaz y la estabilidad de la aplicación durante las pruebas muestran que Flutter es una herramienta adecuada para construir proyectos interactivos y visualmente claros.

La inclusión de objetos como funcionalidad adicional enriqueció la dinámica del combate, permitiendo al usuario tomar decisiones más variadas y agregando un elemento táctico al flujo del juego. Este aspecto permitió explorar temas adicionales como el manejo de estados internos, el control del flujo del combate y la vinculación entre interfaz y lógica de programación.

En general, el proyecto permitió reforzar habilidades de diseño, análisis y desarrollo de software, además de fomentar la creatividad al adaptar las mecánicas del videojuego original a un entorno propio.
