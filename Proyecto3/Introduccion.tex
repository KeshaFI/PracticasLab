El desarrollo de aplicaciones móviles se ha convertido en una de las áreas más dinámicas y relevantes dentro de la ingeniería de software. A lo largo del curso se han revisado conceptos fundamentales de programación, diseño de interfaces y lógica de sistemas, utilizando Flutter como marco de trabajo. Como proyecto integrador, se plantea la creación de una aplicación que simule el sistema de batallas del videojuego Pokémon, con el fin de aplicar de manera práctica los conocimientos adquiridos y demostrar la capacidad de diseñar un sistema interactivo, modular y funcional.

\subsection{Planteamiento del problema}

Los videojuegos de rol y estrategia, como Pokémon, se caracterizan por sistemas de combate basados en turnos, donde las decisiones del jugador y las características de los personajes determinan el resultado de la batalla. Replicar este tipo de mecánicas en una aplicación móvil implica enfrentar diversos retos:

\begin{itemize}
    \item Implementar un sistema de turnos que respete las reglas de prioridad y finalización de la batalla.
    \item Representar las ventajas y desventajas de los tipos de ataque, siguiendo la lógica de daño doble, mitad o nulo.
    \item Diseñar un menú interactivo que muestre al Pokémon del usuario, al rival y las opciones de ataque disponibles.
\end{itemize}

\subsection{Motivación}
La motivación principal de este proyecto es aplicar de manera práctica los conceptos aprendidos en el curso y demostrar cómo Flutter puede ser utilizado para desarrollar aplicaciones interactivas más allá de las tradicionales interfaces de usuario. Recrear un sistema de batallas Pokémon no solo resulta atractivo por su popularidad y valor lúdico, sino que también representa un desafío técnico que fomenta:

\begin{itemize}
    \item El pensamiento lógico y la correcta estructuración de algoritmos.
    \item La La modularidad y reutilización de código, al definir clases y métodos que representen Pokémon, ataques y estados.
    \item El trabajo colaborativo, ya que cada miembro del equipo desarrolla una parte específica del proyecto (objetivos, diagramas UML, desarrollo paso a paso, resultados).
    \item La creatividad y motivación personal, al integrar elementos adicionales como música y estados alterados que enriquecen la simulación.
\end{itemize}

\subsection{Objetivos}
\begin{itemize}
    \item Implementar atributos básicos de los Pokémon: vida (HP) y velocidad.
    \item Diseñar un sistema de turnos que determine el orden de ataque según la velocidad.
    \item Definir al menos un Pokémon y un ataque de cada tipo, respetando las reglas de efectividad (doble, mitad, nulo).
    \item Construir un menú interactivo que muestre los Pokémon en combate y las opciones de ataque.
\end{itemize}

